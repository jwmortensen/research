\documentclass[10pt, compress]{beamer}
\usetheme[titleprogressbar]{m}

\usepackage{booktabs}
\usepackage[scale=2]{ccicons}
\usepackage{minted}
\usepackage{dsfont}
\usepackage{caption}
\captionsetup[figure]{labelformat=empty}
\captionsetup{skip=0pt, belowskip=0pt}

\usepgfplotslibrary{dateplot}

\usemintedstyle{trac}

\title{Poisson Processes and 911 Calls}
\date{\today}
\author{Jacob Mortensen\\ Advisor: Dr. Matt Heaton}
\institute{Brigham Young University}

\begin{document}
  \maketitle
  
  \section{Introduction}
  \begin{frame}
    \frametitle{Introduction}
    \centering
    \includegraphics[width=0.8\textwidth]{houston_map.jpg}
  \end{frame}
  \begin{frame}{Introduction}
    Data: 1389 heat-related 911 calls, 2006-2010 \newline
    \newline
    How many 911 calls will come in and where they will come from? \newline

    \centering
    \begin{minipage}{0.8\textwidth}
      Spatial data is:
      \begin{itemize}
        \item nonlinear
        \item highly correlated
      \end{itemize}
    so how do we model it?
    \end{minipage}
  \end{frame}

  \section{Model}
  \begin{frame}
    \frametitle{Model}
    We use a Poisson point process model
      \begin{itemize}
        \item $N \sim Pois(\int_{\mathcal{S}} \Lambda(s))$
        \item $f(\mathbf{s_1}, \mathbf{s_2}, \dots, \mathbf{s_N}|N) = \prod_{i=1}^{N} \frac{\Lambda(\mathbf{s_i})}{\int_{\mathcal{S}}\Lambda(\mathbf{s})}$ 
        \item $\Lambda(\mathbf{s})$ is the \textbf{intensity function}
        \item $\mathbf{s_1}, \mathbf{s_2}, \dots, \mathbf{s_N}$ where N = 1389 are the latitude-longitude coordinates of the observed 911 calls
        \item $\mathcal{S}$ is the set of all latitude-longitude coordinates in Houston
      \end{itemize}
    \pause
    $\Lambda(\mathbf{s})$ is the key!
  \end{frame}
  \begin{frame}
    \frametitle{Model}
    It can be shown the likelihood is:
      $$L(\Lambda(\mathbf{s})) = \exp\left(-\int_{\mathcal{S}}\Lambda(\mathbf{s})\right) \prod_{i=1}^{N}\Lambda(\mathbf{s}_i)$$
  \end{frame}
  \begin{frame}
    \frametitle{Model}
    \begin{itemize}
      \item To simplify we are going to divide $\mathcal{S}$ into a grid
      \item $\mathbf{s_1}^{*}, \mathbf{s_2}^{*}, \dots, \mathbf{s_K}^{*}$ where K = 1428 are the latitude-longitude coordinates of the  prediction locations
      \item This allows us to discretize $\Lambda(\mathbf{s})$ to make modeling easier:
    \end{itemize}
        $$ \Lambda(\mathbf{s}) = \delta \sum_{k=1}^K \frac{\lambda_k}{|\mathcal{G}_k|} \mathds{1} \left\{ \mathbf{s} \in \mathcal{G}_k \right\} $$

    where $\lambda_k \in (0,1) \text{ and }\sum_{k=1}^K \lambda_k = 1$
  \end{frame}
  \begin{frame}
    \frametitle{Model}
    \centering
    $\Lambda(\mathbf{s}) = \delta \sum_{k=1}^K \frac{\lambda_k}{|\mathcal{G}_k|} \mathds{1} \left\{ \mathbf{s} \in \mathcal{G}_k \right\}$
    \includegraphics[height=0.8\textheight]{blank_grid.pdf}
  \end{frame}
  \begin{frame}
    \frametitle{Model}
    \centering
    $ \Lambda(\mathbf{s}) = \delta \sum_{k=1}^K \frac{\lambda_k}{|\mathcal{G}_k|} \mathds{1} \left\{ \mathbf{s} \in \mathcal{G}_k \right\} $
    \includegraphics[height=0.8\textheight]{grid_spot.pdf}
  \end{frame}
  \begin{frame}
    \frametitle{Model}
    \centering
    $ \Lambda(\mathbf{s}) = \delta \sum_{k=1}^K \frac{\lambda_k}{|\mathcal{G}_k|} \mathds{1} \left\{ \mathbf{s} \in \mathcal{G}_k \right\} $
    \includegraphics[height=0.8\textheight]{grid_spot_highlighted.pdf}
  \end{frame}
  \begin{frame}
    \frametitle{Model}
    With $$\Lambda(\mathbf{s}) = \delta \sum_{k=1}^K \frac{\lambda_k}{|\mathcal{G}_k|} \mathds{1} \left\{ \mathbf{s} \in \mathcal{G}_k \right\}$$
    the likelihood can be rewritten as
    \begin{align*}
      L(\Lambda(\mathbf{s})) &= \exp\left(-\delta\right) \delta^N \prod_{i=1}^{N}\frac{\lambda_{\left\{k: \mathbf{s_i} \in \mathcal{G}_k \right\}}}{|\mathcal{G}_k|} \\
      &= \exp\left(-\delta\right) \delta^N \prod_{i=1}^{K}\frac{\lambda_k^{N_k}}{|\mathcal{G}_k|} 
    \end{align*}
  \end{frame}
  \section{Fitting the Model}
  \begin{frame}
    \frametitle{Fitting the Model}
      $$\delta \sim \text{gamma}(0.001, 0.001)\text{,  conjugate}$$
  \end{frame}
  \begin{frame}
    \frametitle{Fitting the Model}
    How can we use the spatial smoothness in $\Lambda(\mathbf{s})$ to estimate $\lambda_1, \dots, \lambda_k$?
    \pause
    \newline
    \newline
    We use a Gaussian process:
    \begin{itemize}
      \item Let $\lambda_k \propto \exp(\lambda_k^{*})$
      \item Let $\boldsymbol{\lambda}^{*} = \begin{pmatrix} \lambda_1^{*} \\ \vdots \\ \lambda_k^{*} \end{pmatrix} \sim \mathcal{N}(\mathbf{0}, \sigma^2\mathbf{M}_{\nu}(D|\phi))$
      \item Then it follows that $\lambda_k = \frac{\exp(\lambda_k^{*})}{\sum_{j=1}^{K}\exp(\lambda_j^{*})}$

    \end{itemize}
  \end{frame}
  \begin{frame}
    \frametitle{Fitting the Model}
      With $\boldsymbol\lambda^{*} \sim \mathcal{N}(0, \sigma^2\mathbf{M})$, 
    $$ \pi(\boldsymbol\lambda^{*})L(\boldsymbol\lambda^{*} | N) \propto 
      \exp\left(-\frac{1}{2\sigma^2}(\boldsymbol\lambda^{*})^{'}\mathbf{M}^{-1}(\boldsymbol\lambda^{*})\right)\prod_{k=1}^{K}\left(\frac{\exp(\lambda_k^{*})}{\sum_{j=1}^{K}\exp(\lambda_j^{*})}\right)^{N_k} $$
  \end{frame}
  \begin{frame}
    \frametitle{Fitting the Model}
    Using a Metropolis within Gibbs sampler, we generate 100,000 draws for $\delta$ and $\boldsymbol\lambda^{*}$
    \begin{figure}
      \begin{minipage}{0.45\textwidth}
        \includegraphics[width=1.0\textwidth]{delta_trace.png}
        \caption{Trace for $\delta$}
      \end{minipage}
      \hfill
      \begin{minipage}{0.45\textwidth}
        \includegraphics[width=1.0\textwidth]{lambda_trace.png}
        \caption{Trace for $\lambda_{707}$}
      \end{minipage}
    \end{figure}
  \end{frame}
  \section{Results}
  \begin{frame}
    \frametitle{Results}
    \centering
    \begin{figure}
      \caption{Posterior Density of $\delta$}
      \includegraphics[height=0.625\textheight]{delta_density.pdf}
    \end{figure}
    Posterior Mean: 1387.65
  \end{frame}
  \begin{frame}
    \frametitle{Results}
   \centering
    \begin{figure}
      \caption{Posterior Density of $\boldsymbol\lambda$}
      \includegraphics[height=0.7\textheight]{posteriorLocationDensity.pdf}
    \end{figure}
  \end{frame}
  \section{Conclusions}
  \begin{frame}
    \frametitle{Conclusions}
    Successfully answered the questions:
    \begin{itemize}
      \item How many calls?
      \item Where are they at?
    \end{itemize}
    \pause
    \textbf{So what?}
  \end{frame}
  \begin{frame}
    \frametitle{Conclusions}
    Next steps:
    \begin{itemize}
      \item Add in a lagged effect for temperature
      \item Analyze demographics (e.g. age, race, etc.)
      \item Merge 911 call map with a similar map for heat-related mortalities
    \end{itemize}
  \end{frame}
  \section{Questions?}
\end{document}
